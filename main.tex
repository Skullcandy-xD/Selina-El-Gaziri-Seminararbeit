\documentclass{article}
\usepackage[a4paper,left=3cm,right=3cm,top=2cm,bottom=4cm,bindingoffset=5mm]{geometry}
\usepackage[utf8]{inputenc}
\usepackage{amsmath}
\usepackage{amsfonts}
\usepackage{amssymb}
\usepackage[ngerman]{babel}
\usepackage{pdfpages}
\usepackage{cancel}
\usepackage{graphicx}
\usepackage{pst-all}
\usepackage{caption}
\usepackage{booktabs}
\usepackage{tabularx}
\usepackage{graphicx}
\usepackage{float}
\title{Seminararbeit über die komplexe Ordnung hinter dem Chaos}
\author{Selina El Gaziri}
\date{2018}

\begin{document}

\maketitle
\newpage
\tableofcontents
\newpage
\section{Grundoperationen der komplexen Zahlen}

\section{Chaos in der Mathematik}

\section{Fraktale und die komplexen Zahlen}

\subsection{Mengen nach der Iterationsvorschrift $z_{n+1}={z_{n}}^2$ }
\text{Abhängig vom Startwert $z_0$ betrachten wir nun die verschiedenen Fälle am Einheitskreis:}
Man berechnet für die Näherung $ z_1={z_0}^2 => z_2={z_1}^2 => ... => z_{n+1}={z_{n}}^2 =>$
\subsubsection{Gefangenenmenge G_z}
\newline \text{Startwert kleiner als 1; } $|z| = x + y \cdot j < 1$ 
\newline

\noindent
\begin{tabularx}{10cm}{X|X|X|X}
z_{n} = & x & y &  $<1$\\
\hline
z_{0} = & 0,60000 & 0,20000 j & = 0,63250\\
\hline
z_{1} = & 0,32000 & 0,24000 j & = 0,40000\\
\hline
z_{2} =  & 0,04480 & 0,15360 j & = 0,16000\\
\hline
z_{3} = & -0,02159 & 0,01376 j & = 0,02560\\
\hline
z_{4} = & 0,00028	& -0,00059 j & = 0,00070\\
\hline
z_{5} = & 0,00000 & 0,00000 j & = 0,00000\\
\hline
z_{6} = & 0,00000 & 0,00000 j & = 0,00000\\
\end{tabularx}
$z_{n+1} \rightarrow 0$
\newline
\justyifying
\newline Die Zahlenfolge konvergiert gegen 0 und verlässt den Einheitskreis nicht , wie man durch die Zeigermultiplikation sehen kann, sind alle $ z_0   \in ]-1;1[$ eine Gefangenenmenge $G_z$.
\subsubsection{Fluchtmenge F_z}
Startwert größer als 1; $|z| = x + y \cdot j > 1$
\newline
\begin{tabularx}{10cm}{X|X|X|X}
z_{n} = & x & y & $> 1$\\
\hline
z_{0} = & 2,10  & 0,60 j & = 2,18\\
\hline
z_{1} = & 4,05 & 2,52 j & = 4,77\\
\hline
z_{2} = & 10,5 & 20,41 j & = 22,75\\
\hline
z_{3} = & -315,61 & 410,37 j & = 517,69\\
\hline
z_{4} = & -68794,48 & -259027,73 j & = 268007,55\\
\end{tabularx}
$z_{n+1} \rightarrow \infty$
\newline
\newline
Für $z_0>1$ divergiert die Zahlenfolge bestimmt gegen unendlich, und "flüchtet" vor dem Einheitskreis, für $z_0  \in ]- \infty ;-1[ \cup ]1;\infty[$ existiert eine Fluchtmenge.
\subsubsection{Juliamenge J_z}
$|z| = x + y \cdot j = 1$
\newline
\begin{tabularx}{10cm}{X|X|X|X}
z_{n} = & x & y & = 1\\
\hline
z_{0} = & 0,60000  & 0,80000 j & = 1\\
\hline
z_{1} = & -0,28000 & 0,96000 j & = 1\\
\hline
z_{2} = & -0,84320 & -0,53760 j & = 1\\
\hline
z_{3} = & 0,42197 & 0,90661 j & = 1\\
\hline
z_{4} = & -0,64388 & 0,76513 j & = 1\\
\hline
z_{5} = & -0,17084	& -0,98530 j & = 1\\
\hline
z_{6} = & -0,94163 & 0,33666 j & = 1\\
\end{tabularx}
$z_{n+1} \rightarrow 1$
\newline
\newline 
Beträgt $z_0$ genau 1, so ist die Zahlenfolge beschränkt auf 1 und bildet hier den Rand des Einheitskreises. So ist die Beschränkte Menge allgemein der Rand zwischen einer Gefangenenmenge und Fluchtmenge.

\subsection{Juliamengen J}
Erweitern wir unser Polynom 2. Grades noch um die Konstante c $ z_{n+1}={z_{n}}^2 +c (c \in \mathbb{C})$ so erhalten wir für jedes c eine bestimmte Randmenge. Diese nach dem Entdecker Gaston Julia Benannten Grenzverläufe, als auch ihr Inneres, heißen Juliamengen und stellen einen Zusammenhang in der Chaotischen Iteration da.
%Betrachten wir die Beschränkte Menge von $z_{n+1}={z_{n}}^2$ etwas genauer so stellen wir fest das
%$(f) =[z \in C:fk(z)9∞]$
\newline Z_{n+1}={Z_{n}}^2 + C 
\newline n \in \mathbb{N}_{0};Z_n,C \in \mathbb{C}
\newline Z_0=0 =>  Z_1 = C

\subsection{Mandelbrotmenge M}
\newline
\text{Benoit 1980,}
$M = [c \in \mathbb{C} / \lim\limits_{n \rightarrow \infty}{|({f_{C}}^{n}(z))|< \infinity ; z_{0}=0  }]$
\newline Ist die Menge aller Punkte C für die Z=Z*Z + C immer innerhalb des kreises mit r =2 um die Ursprung leid egt 
\newline z \rightarrow z2+c 
\newline wobei\,die\, Iteration\, bei\, dem\, Ursprung\, z_{0}=0\, der Komplexen\, Zahlenebene\, startet\, und\, ist\, die\, Menge\, die\, nicht\, gegen\, \infinity streben


G_{C} = [z0 \in C | z0 \rightarrow z02 + c \rightarrow ... ist beschränkt]

\subsection{Logistische Gleichung und das Feigenbaumdiagramm}

\subsection{Graphische Darstellung}

\subsubsection{Iterationstiefen}

\subsubsection{Apfelmännchen}

\subsubsection{Juliamengen}

\end{document}
